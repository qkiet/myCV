%%%%%%%%%%%%%%%%%%%%%%%%%%%%%%%%%%%%%%%%%
% Medium Length Professional CV
% LaTeX Template
% Version 2.0 (8/5/13)
%
% This template has been downloaded from:
% http://www.LaTeXTemplates.com
%
% Original author:
% Rishi Shah
%
% Important note:
% This template requires the resume.cls file to be in the same directory as the
% .tex file. The resume.cls file provides the resume style used for structuring the
% document.
%
%%%%%%%%%%%%%%%%%%%%%%%%%%%%%%%%%%%%%%%%%

%----------------------------------------------------------------------------------------
%	PACKAGES AND OTHER DOCUMENT CONFIGURATIONS
%----------------------------------------------------------------------------------------

\documentclass{resume} % Use the custom resume.cls style

\usepackage[left=0.75in,top=0.6in,right=0.75in,bottom=0.6in]{geometry} % Document margins
\usepackage[utf8]{vietnam}
\usepackage{hyperref}
\newcommand{\tab}[1]{\hspace{.2667\textwidth}\rlap{#1}}
\newcommand{\itab}[1]{\hspace{0em}\rlap{#1}}
\name{Phạm Quang Kiệt} % Your name
\jobtitle {Embedded software engineer}
\address{} % Your address
%\address{123 Pleasant Lane \\ City, State 12345} % Your secondary addess (optional)
\address{(+84) 933802542 | qkiet97@gmail.com} % Your phone number and email

\begin{document}

%----------------------------------------------------------------------------------------
%	SUMMARY SECTION
%----------------------------------------------------------------------------------------
\begin{rSection}{Summary}
A senior developer with the love of learning and great wealth of knowledge and experience. With \textbf{4+ year} of continuously learning adapt to new technologies and solve new problem, I'm confident that I can provide valuable contribution to teams and organization.
\end{rSection}
%----------------------------------------------------------------------------------------
%	WORK EXPERIENCE SECTION
%----------------------------------------------------------------------------------------


\begin{rSection}{Work Experience}
    \begin{rCompanySubsection}
        {Ban Vien Corporation}
        {07/2022 - 12/2023 (1.5 year)}
        {Senior Embedded Engineer}
        {}
        % Axon BWC project
        \begin{rProjectSubsubsectionV2}{Axon Body Worn Camera (BWC) contractor}{04/2024 - 12/2024}{
            \begin{itemize}
                \item Implement end-to-end automation test for many features of Axon products like Axon Body 3 or Axon Body 4
                \item Improve and expand in-house automation test framework
                \item Maintain Axon's complex application layer
            \end{itemize}
        }{
            \begin{itemize}
                \item Able to fake a RFID scan app in order to perform automation test that use RFID scan app. This problem is solved thanks to browser developer tools, which I am able to extract API and its data
                \item Add to automation test framework Multi-Authentication Factor (MFA) utility to test MFA feature. This problem is solved thanks to, again, browser developer tools and investigation into how MFA really works
                \item Add a sub-command that allows developer to capture what text fields and values LCD is currently display. This feature opens new possibility for automation test because automation test now can check the texts on LCD
                \item Help team move forward by fixing a C++ unit test that weirdly passed in local environment but failed when run in Jenkins environment (spoiler alert: \textit{uninitialized mutex})
                \item Fix a complex issues that involves a lot of processes and happens ocassionally. This problem is interesting because it teaches me how fragile complex software can be and technique to lessen the chance of causing regression.
            \end{itemize}
        }
        \end{rProjectSubsubsectionV2}

        % BV in-house In-vehicle Infotainment project
        \begin{rProjectSubsubsectionV2}{Automative Android In-vehicle Infotainment}{12/2023 - 03/2024 (4 months)}{
            \begin{itemize}
                \item Develop Android App to stream videos produced by ADAS ECU to head Unit
                \item Customize various component in Android Open Source Project (AOSP) to support additional features
            \end{itemize}
        }{
            \begin{itemize}
                \item Able to put the Android streaming video app into launcher (home page) of Android
                \item Pioneer in deeply investigate into various Android platform development topics, most notably are: build system, HAL and VHAL, customized SDK, platform configuration, porting 3rd party library, making digital cluster app. All of this investigation are made into comprehensive documents for team.
                \item Using knowledge of Android build system to port 3rd party library OpenDDS to AOSP, which does not exist in AOSP.
                \item Using knowledge of HAL and VHAL + making digital cluster app to make a POC digital cluster app, whose data is connected to a VHAL back-end service
            \end{itemize}
        }
        \end{rProjectSubsubsectionV2}

        \begin{rProjectSubsubsection}{ION Mobility}{04/2023 - 11/2023 (8 months)}
            \begin{itemize}
                \item Support ION Mobility team to integrate AUTOSAR diagnostic module DEM, DET into ION's ECU codebase
                \item Develop unique WiFi solution for ION's ECU which does not have built-in WiFi module
                \item Develop MQTT drivers for ION's ECU to communicate with cloud
                \item Develop AT command driver in ION's ECU to seamlessly transmit AT command to LTE modem and invoke receive callback
            \end{itemize}
            \hspace*{2.5em}Technologies: \textbf{WiFi, LTE, MQTT, AUTOSAR}
        \end{rProjectSubsubsection}
        \begin{rProjectSubsubsection}{Upstream Kernel Verification}{01/2023 - 03/2023 (3 months)}
            \begin{itemize}
                \setlength{\itemindent}{1.25em}
                \item Report changes of maintained drivers from latest version of mainline kernel
                \item Perform regression test on maintained drivers to make sure they are functional
                \item Record, attempt to fix and report any failutres of regression test
            \end{itemize}
            \hspace*{2.5em}Technologies: \textbf{Linux Device Driver, Test Automation}
        \end{rProjectSubsubsection}
        \begin{rProjectSubsubsection}{Advanced Driver-Assistance System}{11/2022 - 12/2022 (2 months)}
            \begin{itemize}
                \setlength{\itemindent}{1.25em}
                \item Create Yocto recipes in order to build Linux disto that:
                \begin{itemize}
                    \item Make Wifi dongle works for the distro
                    \item Make Wifi dongle act like a Wifi access point after system boots
                    \item Integrate in-house software into the distro
                \end{itemize}
            \end{itemize}
            \hspace*{2.5em}Technologies: \textbf{Yocto, systemd, WiFi, Linux Kernel}
        \end{rProjectSubsubsection}
        \begin{rProjectSubsubsection}{Linux Kernel BSP}{08/2022 - 03/2022 (5 months)}
            \begin{itemize}
                \setlength{\itemindent}{1.25em}
                \item Help the team fix Ethernet and PCIe-related bugs in Linux kernel
                \item Some noteable issues include:
                \begin{itemize}
                    \item Boards not able to ping even physical connection is established
                    \item PCIe only works with one host
                    \item PCIe not able to transfer at full speed
                \end{itemize}
            \end{itemize}
            \hspace*{2.5em}Technologies: \textbf{PCIe, Ethernet, DMA, Linux Device Driver}
        \end{rProjectSubsubsection}
    \end{rCompanySubsection}
\begin{rCompanySubsection}
    {XelexBanvien Corporation}
    {017/20212 - 0612/2022 (1.5 year)}
    {Firmware Engineer}
    {}

\begin{itemize}
     \item Coordinate with hardware team for \textbf{schematic revision and recommendation} to make sure defined features are functional
     \item \textbf{Develop, monitor and adjust firmware development plan} to make sure projects meet its deadline
     \item \textbf{Manage project documentation database} to make sure documentation database is easy to navigate and new recruits can understand project structure
     \item Customize \textbf{MEC1428 (a M14K core MCU)} source code to aid CPU power sequence and control \& monitor platform's hardware
     \item \textbf{Customize BIOS}, which uses {\bf Slim bootloader} framework, to make sure platform is functional as expected. Customization include \textbf{reconfigure GPIO}, \textbf{configure SoC supported features} and customize \textbf{ACPI firmware} (to make sure Windows driver recognize and configure hardware)
     \item Debug prototypes using following methods: \textbf{logs} produced from BIOS; probing \textbf{power \& data signals} using \textit{multimeter, oscilloscope and logic analyzer}; using \textbf{Windows tools} to inspect hardware communication
     \item \textbf{Work closely with suppliers} for firmware support and request additional tools \& documents

\end{itemize}
\end{rCompanySubsection}

\begin{rCompanySubsection}
    {Xelex Corporation}
    {03/2020 - 12/2020 (9 months)}
    {Firmware Engineer Intern}
    {}
\begin{itemize}
     \item Define {\bf microarchitecture-based firmware architecture, inter-thread communication flow, data flow, security protocol} to accommodate complex system features
     \item \textbf{Define customized communication \& security protocol} between multiple IoT devices and IoT device with management software in host computer base on \textbf{HTTP} to facilitate secured data-and-control transmission
     \item Develop features defined in architectures. MCU is {\bf ARM-based STM32F407}, develop with {\bf STM32CubeIDE}, then migrated to {\bf Visual Code + Cmake} to incorporate {\bf Git} in develop environment and ease develop flow. Embedded OS is {\bf FreeRTOS} (a  {\bf RTOS MIT-licensed implementation} which suitable for corporation project)
     \item {\bf Assign, monitor tasks and guide} team members to have team member meet project's technical knowledge and make sure team effort is aligned with the project.

\end{itemize}
\end{rCompanySubsection}




\end{rSection}
\begin{rSection}{Education}

{\bf University of Technology Ho Chi Minh City} \hfill {\em 08/2015 - 2021}
\\ Electronics - Telecommunications Engineering\hfill { GPA: 7.33/10 }


\textbf{\textit{Thesis:}} Develop a new \textbf{lightweight security protocol} that used for communication between Power Distribution Unit (PDU) and computer software. This security protocol features:
\begin{itemize}
    \item Encrypted transmitted messages with \textbf{non-repeat encrypt key}. Encrypt algorithm using \textbf{AES-CTR-128 for utilizing parallel computation}, thus reduce computation time.
    \item Detect unauthorized modification of incoming messages using \textbf{Hash-based Message Authentication Code (HMAC)}.
    \item \textbf{Customized encrypt key exchange protocol} features 3-way handshake key exchange and emergency procedure to restore encrypt key for both sides in case of corrupted message.
\end{itemize}
Implementation of this protocol including utilizing \textbf{lwIP + FreeRTOS} on PDU's firmware and \textbf{Visual C++} on computer host software.\\
Target device firmware source code: \href{https://github.com/qkiet/Test_Security_System}{https://github.com/qkiet/Test\_Security\_System}\\
Host computer software source code:
\href{https://github.com/qkiet/Demo_Monitor_Control}{https://github.com/qkiet/Demo\_Monitor\_Control}\
%Minor in Linguistics \smallskip \\
%Member of Eta Kappa Nu \\
%Member of Upsilon Pi Epsilon \\


\end{rSection}
% \begin{rSection}{Academic Projects}
% {\bf Smart Alarm controlled by BLE}\hfill {\em 12/2018}
% \begin{itemize}
%     \item Design a system that alarm when detect vibration
%     \item Using a Bluetooth Low Energy to setup device operation mode
%     \item \textbf{Roles:} Program the microcontroller to detect vibration and setup operation mode with BLE message
%     \item \textbf{Result:} Scored by instructor: \textbf{9.0/10}

% \end{itemize}

% {\bf Traffic sign recognition by video}\hfill {\em 05/2019}
% \begin{itemize}
%     \item Design an algorithm to detect traffic signs in video and recognize the name of the traffic signs
%     \item \textbf{Roles:} Tuning detection algorithm's setting and extract features from detected Region of Interest that contain traffic signs
%     \item \textbf{Result:} Scored by instructor: \textbf{8.5/10}


% \end{itemize}

% {\bf Measure heart rate by capture your face using camera }\hfill {\em 12/2019}
% \begin{itemize}
%     \item Design an algorithm that measure instantaneous heart rate by utilizing a video that capture human face
%     \item \textbf{Roles:} Project leader, rewrite heart rate measurement algorithm from referenced sources, program some miscellaneous features to speedup testing and visualizing signals. Organize tasks for team member to finish final reports
%     \item \textbf{Result} Scored by instructor: \textbf{9.0/10}

% \end{itemize}
% {\bf Collect shrimp pond's parameters and transmit by LoRa technology}\hfill {\em 01/2020}
% \begin{itemize}
%     \item Design a system with two modules: the transmitter collects shrimp pond's parameters like temperature, pH... then transmit parameters to receiver by LoRa module; the receiver receives those parameters then upload to Thingspeak.
%     \item Project contains an Android application to oversee shrimp pond's parameters by download data from Thingspeak
%     \item \textbf{Roles:} Program microcontrollers in transmitter and receiver, program Android application, design circuit layout and soldier circuit.
%     \item \textbf{Result} Scored by instructor: \textbf{9.5/10}
% \end{itemize}

% \end{rSection}


%--------------------------------------------------------------------------------
%    Projects And Seminars
%-----------------------------------------------------------------------------------------------

%----------------------------------------------------------------------------------------
%	KEY SKILLS SECTION
%----------------------------------------------------------------------------------------

\begin{rSection}{Key Skills}

\begin{tabular}{ @{} >{\bfseries}l @{\hspace{6ex}} l }
Languages & C, C++, Java, Kotlin, C\#, Python, Golang \\
Technologies & RTOS, EDK II, ACPI, Jenkins, Docker, AOSP, Yocto
% Integrated Development Environment & Visual Studio, Arduino IDE \\
% Circuit Development Tools & Proteus, Advanced Design System\\
\end{tabular}

\end{rSection}
\begin{rSection}{Language} \itemsep -3pt
\item TOEIC \textbf{915/990}

\end{rSection}


%----------------------------------------------------------------------------------------
%	WORK EXPERIENCE SECTION
%----------------------------------------------------------------------------------------




%	EXAMPLE SECTION
%----------------------------------------------------------------------------------------

\end{document}
