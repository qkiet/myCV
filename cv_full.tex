%%%%%%%%%%%%%%%%%%%%%%%%%%%%%%%%%%%%%%%%%
% Medium Length Professional CV
% LaTeX Template
% Version 2.0 (8/5/13)
%
% This template has been downloaded from:
% http://www.LaTeXTemplates.com
%
% Original author:
% Rishi Shah
%
% Important note:
% This template requires the resume.cls file to be in the same directory as the
% .tex file. The resume.cls file provides the resume style used for structuring the
% document.
%
%%%%%%%%%%%%%%%%%%%%%%%%%%%%%%%%%%%%%%%%%

%----------------------------------------------------------------------------------------
%	PACKAGES AND OTHER DOCUMENT CONFIGURATIONS
%----------------------------------------------------------------------------------------

\documentclass{resume} % Use the custom resume.cls style

\usepackage[left=0.75in,top=0.6in,right=0.75in,bottom=0.6in]{geometry} % Document margins
\usepackage[utf8]{vietnam}
\usepackage{hyperref}
\newcommand{\tab}[1]{\hspace{.2667\textwidth}\rlap{#1}}
\newcommand{\itab}[1]{\hspace{0em}\rlap{#1}}
\name{Phạm Quang Kiệt} % Your name
\jobtitle {Embedded software engineer}
\address{} % Your address
%\address{123 Pleasant Lane \\ City, State 12345} % Your secondary addess (optional)
\address{(+84) 933802542 | qkiet97@gmail.com} % Your phone number and email

\begin{document}

%----------------------------------------------------------------------------------------
%	SUMMARY SECTION
%----------------------------------------------------------------------------------------
\begin{rSection}{Summary}
A senior developer with the love of learning and great wealth of knowledge and experience. With \textbf{4+ year} of continuously learning adapt to new technologies and solve new problem, I'm confident that I can provide valuable contribution to teams and organization.
\end{rSection}
%----------------------------------------------------------------------------------------
%	WORK EXPERIENCE SECTION
%----------------------------------------------------------------------------------------


\begin{rSection}{Work Experience}
    \begin{rCompanySubsection}{Ban Vien Corporation}{07/2022 - 12/2024 (2.5 year)}
        {Senior Embedded Engineer}
        {}
        % Axon BWC project
        \begin{rProjectSubsubsectionV2}{Axon Body Worn Camera (BWC) contractor}{04/2024 - 12/2024}{
            \begin{itemize}
                \item Implement end-to-end automation test for many features of Axon products like Axon Body 3 or Axon Body 4
                \item Improve and expand in-house automation test framework
                \item Maintain Axon's complex application layer
            \end{itemize}
        }{
            \begin{itemize}
                \item Able to fake a RFID scan app in order to perform automation test that use RFID scan app. This problem is solved thanks to browser developer tools, which I am able to extract API and its data
                \item Add to automation test framework Multi-Authentication Factor (MFA) utility to test MFA feature. This problem is solved thanks to, again, browser developer tools and investigation into how MFA really works
                \item Add a sub-command that allows developer to capture what text fields and values LCD is currently display. This feature opens new possibility for automation test because automation test now can check the texts on LCD
                \item Help team move forward by fixing a C++ unit test that weirdly passed in local environment but failed when run in Jenkins environment (spoiler alert: \textit{uninitialized mutex})
                \item Fix a complex issues that involves a lot of processes and happens ocassionally. This problem is interesting because it teaches me how fragile complex software can be and technique to lessen the chance of causing regression.
            \end{itemize}
        }
        \end{rProjectSubsubsectionV2}

        % BV in-house In-vehicle Infotainment project
        \begin{rProjectSubsubsectionV2}{Automative Android In-vehicle Infotainment}{12/2023 - 03/2024 (4 months)}{
            \begin{itemize}
                \item Develop Android App to stream videos produced by ADAS ECU to head Unit
                \item Customize various component in Android Open Source Project (AOSP) to support additional features
            \end{itemize}
        }{
            \begin{itemize}
                \item Able to put the Android streaming video app into launcher (home page) of Android
                \item Pioneer in deeply investigate into various Android platform development topics, most notably are: build system, HAL and VHAL, customized SDK, platform configuration, porting 3rd party library, making digital cluster app. All of this investigation are made into comprehensive documents for team.
                \item Using knowledge of Android build system to port 3rd party library OpenDDS to AOSP, which does not exist in AOSP.
                \item Using knowledge of HAL and VHAL + making digital cluster app to make a POC digital cluster app, whose data is connected to a VHAL back-end service
            \end{itemize}
        }
        \end{rProjectSubsubsectionV2}

        % ION Mobility project
        \begin{rProjectSubsubsectionV2}{ION Mobility}{04/2023 - 11/2023 (7 months)}{
            \begin{itemize}
                \item Support implementing AUTOSAR various module like DET and DEM
                \item Develop additional complex driver in order to bring up features like WiFi
            \end{itemize}
        }{
            \begin{itemize}
                \item Fix an buffer overflow bug: sometimes UART data transmitting from one ECU to another is dropped unexpectedly. Turns out the receive handling doing too much processing (like checksum comparision) that leads to data loss. Counter this issue by forwarding UART data directly to a circular queue for later processing.
                \item Develop a unique WiFi solution for ION's ECU which does not have built-in WiFi module but have external ESP-32 module. This is accomplished by porting ESP Hosted into ECU (a complex device driver CDD) and ESP-32 module. A challenge is: I must find the workaround for mutex because ESP Hosted in MCU side uses a lot of mutex but platform realtime OS OSEK does not support mutex.
                \item Develop a CDD called Wifi Manager to manage Wi-Fi connection. This driver is implemented by using my own finite state machine implementation and leveraging above Wi-Fi driver.
                \item Develop a CDD called LTE driver to seamlessly communicate with LTE module to make HTTP request or using built-in MQTT AT command
                \item Develop a CDD called MQTT driver that manage MQTT connection, get published data and push data. This MQTT driver is designed to be used by multiple transport: by either Wi-Fi or LTE
            \end{itemize}
        }
        \end{rProjectSubsubsectionV2}

        % RVC UKV
        \begin{rProjectSubsubsectionV2}{RVC Upstream Kernel Verification}{01/2023 - 03/2023 (3 months)} {
            \begin{itemize}
                \item Report changes of maintained drivers of Reneas Design Vietnam Co,.Ltd (RVC) from latest version of mainline kernel
                \item Perform regression test of maintained drivers on RCAR-H3 and RCAR-M3 boards
                \item Record, attempt to fix and report any failutres of regression test
            \end{itemize}
        }{
            \begin{itemize}
                \item Inspired from Yocto project, I implement a custom build system that automatically fetch, configure and build kernel
            \end{itemize}
        }
        \end{rProjectSubsubsectionV2}

        % BV in-house ADAS
        \begin{rProjectSubsubsectionV2}{In-house ADAS}{11/2022 - 12/2022 (2 months)} {
            \begin{itemize}
                \item Support various features for proof-of-concept ADAS platform
            \end{itemize}
        }{
            \begin{itemize}
                \item Integrate Wi-Fi AP service, which ADAS platform serves as a Wi-Fi AP, via a Wi-Fi USB dongle into ADAS platform distros. This problem is more challenge than it is previously thought because I have to make the Wi-Fi dongle works on Linux distros of ADAS platform, investigate how to make a STA mode Wi-Fi dongle into AP mode and make systemd service in Yocto Project
            \end{itemize}
        }
        \end{rProjectSubsubsectionV2}

        \begin{rProjectSubsubsectionV2}{RVC Linux Kernel BSP}{08/2022 - 03/2022 (5 months)} {
        }{
            \begin{itemize}
                \item Help the team fix Ethernet and PCIe-related bugs in Linux kernel for new RCAR-S4 board
            \end{itemize}
        }{
            \begin{itemize}
                \item Fix issues that PCIe cannot work in one of channel thanks to deep investigation in RCAR-S4 Hardware User Manual
            \end{itemize}
        }
        \end{rProjectSubsubsectionV2}
    \end{rCompanySubsection}
    \begin{rCompanySubsection}{Xelex Corporation}{01/2021 - 06/2022 (1.5 year)}
        {Firmware Engineer}
        {}
        \begin{rProjectSubsubsectionV2}{Build BIOS for new Window-powered tablet}{04/2024 - 12/2024}{
            \begin{itemize}
                \item Build BIOS and various firmware in order to bring up features of the tablet
                \item Work closely with other team to make sure the design can meet its objective
            \end{itemize}
        }{
            \begin{itemize}
                \item Able to bring up Intel Boot Guard feature, which is crucial for boot security
                \item Able to bring up on-platform Trusted Platform Module for TPM depedent features
                \item Able to bring up fingerprint thanks to configure ACPI tables
            \end{itemize}
        }
        \end{rProjectSubsubsectionV2}
    \end{rCompanySubsection}

    \begin{rCompanySubsection} {Xelex Corporation} {03/2020 - 12/2020 (9 months)}
        {Firmware Engineer Intern}
        {}
        \begin{rProjectSubsubsectionV2}{Build firmware for Smart PDU product}{04/2024 - 12/2024}{
            \begin{itemize}
                \item Design Smart PDU firmware architecture
                \item Implement Smart PDU firmware to meet all defined features
                \item Assign and monitor tasks of team member
            \end{itemize}
        }{
            \begin{itemize}
                \item \textbf{Define customized communication \& security protocol} between multiple IoT devices and IoT device with management software in host computer base on \textbf{HTTP} to facilitate secured data-and-control transmission
            \end{itemize}
        }
        \end{rProjectSubsubsectionV2}
    \end{rCompanySubsection}
\end{rSection}
\begin{rSection}{Education}

{\bf University of Technology Ho Chi Minh City} \hfill {\em 08/2015 - 2021}
\\ Electronics - Telecommunications Engineering\hfill { GPA: 7.33/10 }


\textbf{\textit{Thesis:}} Develop a new \textbf{lightweight security protocol} that used for communication between Power Distribution Unit (PDU) and computer software. This security protocol features:
\begin{itemize}
    \item Encrypted transmitted messages with \textbf{non-repeat encrypt key}. Encrypt algorithm using \textbf{AES-CTR-128 for utilizing parallel computation}, thus reduce computation time.
    \item Detect unauthorized modification of incoming messages using \textbf{Hash-based Message Authentication Code (HMAC)}.
    \item \textbf{Customized encrypt key exchange protocol} features 3-way handshake key exchange and emergency procedure to restore encrypt key for both sides in case of corrupted message.
\end{itemize}
Implementation of this protocol including utilizing \textbf{lwIP + FreeRTOS} on PDU's firmware and \textbf{Visual C++} on computer host software.\\
Target device firmware source code: \href{https://github.com/qkiet/Test_Security_System}{https://github.com/qkiet/Test\_Security\_System}\\
Host computer software source code:
\href{https://github.com/qkiet/Demo_Monitor_Control}{https://github.com/qkiet/Demo\_Monitor\_Control}\
%Minor in Linguistics \smallskip \\
%Member of Eta Kappa Nu \\
%Member of Upsilon Pi Epsilon \\


\end{rSection}
% \begin{rSection}{Academic Projects}
% {\bf Smart Alarm controlled by BLE}\hfill {\em 12/2018}
% \begin{itemize}
%     \item Design a system that alarm when detect vibration
%     \item Using a Bluetooth Low Energy to setup device operation mode
%     \item \textbf{Roles:} Program the microcontroller to detect vibration and setup operation mode with BLE message
%     \item \textbf{Result:} Scored by instructor: \textbf{9.0/10}

% \end{itemize}

% {\bf Traffic sign recognition by video}\hfill {\em 05/2019}
% \begin{itemize}
%     \item Design an algorithm to detect traffic signs in video and recognize the name of the traffic signs
%     \item \textbf{Roles:} Tuning detection algorithm's setting and extract features from detected Region of Interest that contain traffic signs
%     \item \textbf{Result:} Scored by instructor: \textbf{8.5/10}


% \end{itemize}

% {\bf Measure heart rate by capture your face using camera }\hfill {\em 12/2019}
% \begin{itemize}
%     \item Design an algorithm that measure instantaneous heart rate by utilizing a video that capture human face
%     \item \textbf{Roles:} Project leader, rewrite heart rate measurement algorithm from referenced sources, program some miscellaneous features to speedup testing and visualizing signals. Organize tasks for team member to finish final reports
%     \item \textbf{Result} Scored by instructor: \textbf{9.0/10}

% \end{itemize}
% {\bf Collect shrimp pond's parameters and transmit by LoRa technology}\hfill {\em 01/2020}
% \begin{itemize}
%     \item Design a system with two modules: the transmitter collects shrimp pond's parameters like temperature, pH... then transmit parameters to receiver by LoRa module; the receiver receives those parameters then upload to Thingspeak.
%     \item Project contains an Android application to oversee shrimp pond's parameters by download data from Thingspeak
%     \item \textbf{Roles:} Program microcontrollers in transmitter and receiver, program Android application, design circuit layout and soldier circuit.
%     \item \textbf{Result} Scored by instructor: \textbf{9.5/10}
% \end{itemize}

% \end{rSection}


%--------------------------------------------------------------------------------
%    Projects And Seminars
%-----------------------------------------------------------------------------------------------

%----------------------------------------------------------------------------------------
%	KEY SKILLS SECTION
%----------------------------------------------------------------------------------------

\begin{rSection}{Key Skills}

\begin{tabular}{ @{} >{\bfseries}l @{\hspace{6ex}} l }
Languages & C, C++, Java, Kotlin, C\#, Python, Golang \\
Technologies & RTOS, EDK II, ACPI, Jenkins, Docker, AOSP, Yocto
% Integrated Development Environment & Visual Studio, Arduino IDE \\
% Circuit Development Tools & Proteus, Advanced Design System\\
\end{tabular}

\end{rSection}
\begin{rSection}{Language} \itemsep -3pt
\item TOEIC \textbf{915/990}

\end{rSection}


%----------------------------------------------------------------------------------------
%	WORK EXPERIENCE SECTION
%----------------------------------------------------------------------------------------




%	EXAMPLE SECTION
%----------------------------------------------------------------------------------------

\end{document}
